\documentclass[10pt, a4paper]{article}
%\usepackage[english]{babel}
\usepackage[brazilian]{babel}
\usepackage[utf8]{inputenc}
% \usepackage[T1]{fontenc}
\usepackage{lipsum}

% code
\usepackage{pythonhighlight}
\renewcommand{\lstlistingname}{Anexo} % Listing->Code
\usepackage{adjustbox}

% For subfigure use
\usepackage[font=small,labelfont=bf]{caption}
\usepackage{subcaption}

% Set page size and margins
% Replace `letterpaper' with`a4paper' for UK/EU standard size
\usepackage[a4paper,top=2cm,bottom=2cm,left=2cm,right=2cm,marginparwidth=2cm]{geometry}

% tabelas
\usepackage{array}
\usepackage{tabularx}
\usepackage{booktabs}

\usepackage{float}

% Useful packages
\usepackage{amsmath}
\usepackage{enumerate}

\usepackage{graphicx}
\usepackage[colorlinks=true, allcolors=blue]{hyperref}
\usepackage{cleveref}
\newcommand{\crefrangeconjunction}{--}


\begin{document}

\def\TITLE{Atividade 01}
\def\DISCIPLINE{INF 2137 - Engenharia de software para ciência de dados}
\def\PROFESSOR{Marcos Kalinowski}
\def\AUTHOR{Pedro Henrique Cardoso Paulo}
\def\CONTACT{pedrorjpaulo.phcp@gmail.com}
\def\DATE{abril, 2023}

\title{\textbf{\TITLE} \\ \DISCIPLINE}
\author{\AUTHOR}
\date{\DATE}

\begin{titlepage}
      \begin{center}
          \vspace*{1cm}

          \Huge
          \textbf{\TITLE}

          \vspace{0.5cm}
          \LARGE
          \DISCIPLINE

          \vspace{1.5cm}

          \textbf{\AUTHOR \\ {\tt \CONTACT}}

          \vfill
          Professor: \PROFESSOR

          \vspace{0.8cm}

          \includegraphics[width=0.2\textwidth]{../general/puc.jpg}

          \Large
          Departamento de Engenharia Mecânica\\
          PUC-RJ Pontifícia Universidade Católica do Rio de Janeiro\\
          \DATE

      \end{center}
  \end{titlepage}

\maketitle

\section{Introdução}

Esta tarefa tem por objetivo exercitar as competências de especificação de softwares nteligentes usando a metodologia do PerSpecML.

\section{Enunciado}

\subsection{Contexto}

Uma empresa fornece uma solução de marketplace em que diferentes anunciantes (sellers) podem anunciar seus produtos para venda. Como diferentes tipos de 
produtos requerem diferentes informações cadastrais (por exemplo, camisas possuem informações como material, tamanho e cor, enquanto telefones celulares possuem 
informações como processador, espaço de armazenamento, capacidade da bateria, entre outros), a empresa deseja implementar uma classificação automática dos tipos 
de produtos com base nos títulos dos anúncios utilizando aprendizado de máquina. Desta forma, ela será capaz de apresentar para o anunciante os formulários 
apropriados para o cadastro das informações detalhadas, maximizando as chances de venda.

A empresa acredita que com essa classificação automática e a consequente melhoria das informações cadastrais ela será capaz de aumentar as vendas do marketplace em 
5\%. Os usuários esperam poder realizar o cadastro de forma prática, sem complicações de classificação manual. Do ponto de vista mais técnico, espera-se que o 
modelo de aprendizado de máquina tenha uma acurácia de pelo menos 80\% na classificação. Além disso, devido ao grande volume de anunciantes, o modelo deverá ser 
capaz de processar pelo menos 10.000 requisições por minuto.

\subsection{User Story}

\textbf{Como} anunciante do marketplace \textbf{quero} obter um formulário com as informações cadastrais corretas a partir do título do meu produto 
\textbf{para} poder cadastrar o produto de forma precisa, maximizando as chances de venda.

\subsection{Tarefa}

Realize a especificação utilizando o template da técnica PerSpecML do Miro disponível neste  \href{https://miro.com/miroverse/perspecml-machine-learning/}{link}. 
Como resolução da tarefa no ambiente, poste um PDF com um link para a sua cópia do board do Miro. Opcionalmente, inclua no PDF qualquer outra informação que julgar 
pertinente, como, por exemplo, feedback sobre a aplicação da técnica.

\section{Solução}

\subsection{Link}

\begin{itemize}
    \item \href{https://miro.com/app/board/uXjVMUQCfBQ=/}{Link para o Miro com a resolução}
\end{itemize}

\subsection{Premissas addotadas e pontos relevantes}

\begin{itemize}
    \item Como não houve requisito explícito de tempo máximo para inferência, foi adotado 5 segundos. O requisito de 10000 operações por minuto foi explicitado também,\ 
    mas este pode ser compensado por meio da paralelização do serviço em múltiplos conteineres.
    \item Foi entendido pela explicação da tarefa que o sistema seria \textit{forceful} e que sertia mais danoso para o usuário ter ma predição errada do que uma não\
    predição. Dessa forma, isso foi explicitado no quadro.
    \item Para garantir algum nível de \textit{feedback} dos cadastradores, assumiu-se que eles teriam uma opção de ignorar a saída do sistema e ir para um formulário\
    genérico. Pelos motivos explicitados na explicação (muitas categorias), eles não poderiam reclassificar manualmente.
    \item Como o marketplace fica aberto à compradores, assumiu-se que compradores poderiam dar o feedback sobre a classificação feita.
    \item Como não sabemos se o marketplace contrata contêineres na nuvem ou usa servidores próprios, ambas as opções foram explicitadas.
\end{itemize}

\subsection{Impressões gerais sobre a técnica}

\begin{itemize}
    \item A aplicação da técnica foi bem bacana pois dá espaço para assuntos bem focados na especificação para softwares inteligentes. Normalmente quando se usa dinâmicas\
    mais genéricas como um \textit{lean canvas} simples, o espaço para divergêncas é muito grande. Para fases em que já se definiu que o produto será uma solução digital\
    focada em ML, a dinâmica é muito boa.
    \item A separação dos tópicos parece ajudar especialistas de cada área a se acharem e contribuir. Pessoalmente, ficou fácil achar os tópicos que eu tinha mais domínio
    \item Não parece ser uma dinâmica fácil de fazer se não tiver um bom grupo envolvido. Pelo próprio exercício, em que fazemos sozinhos, fica evidente que falta a\
    presença de mais gente que conheça do problema para detalhar algumas áreas.
\end{itemize}






%%%%%%%%%%%%%%%%%%%%%%%%%%%%%%%%%%%%%%%%%%%%%%%%%%%

\bibliographystyle{apalike}
\bibliography{export}

\end{document}