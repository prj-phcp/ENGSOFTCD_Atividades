\documentclass[10pt, a4paper]{article}
%\usepackage[english]{babel}
\usepackage[brazilian]{babel}
\usepackage[utf8]{inputenc}
% \usepackage[T1]{fontenc}
\usepackage{lipsum}

% code
\usepackage{pythonhighlight}
\renewcommand{\lstlistingname}{Anexo} % Listing->Code
\usepackage{adjustbox}

% For subfigure use
\usepackage[font=small,labelfont=bf]{caption}
\usepackage{subcaption}

% Set page size and margins
% Replace `letterpaper' with`a4paper' for UK/EU standard size
\usepackage[a4paper,top=2cm,bottom=2cm,left=2cm,right=2cm,marginparwidth=2cm]{geometry}

% tabelas
\usepackage{array}
\usepackage{tabularx}
\usepackage{booktabs}

\usepackage{float}

% Useful packages
\usepackage{amsmath}
\usepackage{enumerate}

\usepackage{graphicx}
\usepackage[colorlinks=true, allcolors=blue]{hyperref}
\usepackage{cleveref}
\newcommand{\crefrangeconjunction}{--}


\begin{document}

\def\TITLE{Atividade 02}
\def\DISCIPLINE{INF 2137 - Engenharia de software para ciência de dados}
\def\PROFESSOR{Marcos Kalinowski}
\def\AUTHOR{Pedro Henrique Cardoso Paulo}
\def\CONTACT{pedrorjpaulo.phcp@gmail.com}
\def\DATE{julho, 2023}

\title{\textbf{\TITLE} \\ \DISCIPLINE}
\author{\AUTHOR}
\date{\DATE}

\begin{titlepage}
      \begin{center}
          \vspace*{1cm}

          \Huge
          \textbf{\TITLE}

          \vspace{0.5cm}
          \LARGE
          \DISCIPLINE

          \vspace{1.5cm}

          \textbf{\AUTHOR \\ {\tt \CONTACT}}

          \vfill
          Professor: \PROFESSOR

          \vspace{0.8cm}

          \includegraphics[width=0.2\textwidth]{../general/puc.jpg}

          \Large
          Departamento de Engenharia Mecânica\\
          PUC-RJ Pontifícia Universidade Católica do Rio de Janeiro\\
          \DATE

      \end{center}
  \end{titlepage}

\maketitle

\section{Introdução}

Esta tarefa tem por objetivo realizar um pequeno projeto de software inteligente usando as boas práticas aprendidas no curso.

\section{Enunciado}

\subsection{Tarefa}

Escolha uma base de dados para um problema de classificação ou regressão. Sugere-se usar uma das bases de dados disponibilizada no UCI 
Machine Learning Repository ou no Kaggle. Você deverá trabalhar desde o problema até os resultados e necessariamente comparar e 
otimizar os hiperparâmetros para mais de um modelo.

Lembre-se das etapas para implementação de projetos de ciência de dados, apresentadas em sala de aula, e trabalhe com 
esta base de dados no ambiente Google Colab, utilizando Python e a biblioteca Scikit-Learn. Produza um notebook que servirá 
como relatório, descrevendo textualmente cada uma das etapas do seu código. Para a definição do problema, descreva uma user story 
e seu detalhamento (simplificado – não é necessário utilizar o template de especificação da atividade anterior, já que é um toy problem 
e não um projeto industrial), na primeira célula do notebook. Para a implementação, sugere-se utilizar as boas práticas de programação 
orientada a objetos trabalhadas em sala.

Na última célula, informe como você implantaria a solução de forma a disponibilizar o resultado do seu modelo para o usuário final.

\subsection{User Story}

\textbf{Como} engenheiro de reservatório \textbf{quero} prever a produção de óleo de um poço baseado em seu histórico 
\textbf{para} tomar decisões de priorização e cumprir obrigações legais.

\section{Solução}

\subsection{Link}

\begin{itemize}
    \item \href{https://github.com/prj-phcp/ENGSOFTCD_Atividades/tree/master/Atividade02}{Pasta geral com o conteúdo e dataset}
    \item \href{https://github.com/prj-phcp/ENGSOFTCD_Atividades/blob/master/Atividade02/Atividade02.ipynb}{Notebook com entrega final}
\end{itemize}


%%%%%%%%%%%%%%%%%%%%%%%%%%%%%%%%%%%%%%%%%%%%%%%%%%%

\bibliographystyle{apalike}
\bibliography{export}

\end{document}